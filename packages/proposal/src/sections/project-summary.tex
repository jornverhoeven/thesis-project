\section{Project Summary}

The exponential growth of the internet has revolutionized several aspects of our modern life, such as communication, entertainment, and the way we work. However, this increasing level of connectivity also brings inherent security risks and breaches. These can compromise sensitive information and cause substantial financial and reputational damage to individuals and organizations. As a result, there is a growing need for effective and possibly automated risk management and network security measures to mitigate these risks.


In recent years, Common Vulnerabilities and Exposures (CVEs) have emerged as a critical component of risk management and network security. CVEs are publicly disclosed vulnerabilities and weaknesses that have been identified in software and hardware products. Each CVE is assigned a unique identification number, and detailed information about the vulnerability is published in a public database maintained by the National Cybersecurity and Communications Integration Center (NCCIC).
\add{Maybe add links to CVE and NCCIC}

By regularly monitoring the CVE database and applying software patches and updates to address known vulnerabilities, organizations can significantly reduce the risk of security breaches and other cyber threats. Failure to address known CVEs can leave systems vulnerable to a range of attacks, including malware infections, data breaches, and denial-of-service (DoS) attacks.

As of today, there are many vulnerability scanning tools available. Through this, any organizations can identify vulnerabilities in their applications and systems and prioritize them based on severity and potential impact. The software provides detailed information about each vulnerability, including its severity and potential impact. This information enables organizations to develop and implement effective mitigation strategies to address the identified vulnerabilities.

While these vulnerability scanners can be powerful tools for identifying and sometimes mitigating vulnerabilities in software applications and systems, they are limited in their ability to identify and address vulnerabilities that may exist beyond a single node of a network.

This is where \textbf{A}utomated \textbf{D}ecentralized \textbf{R}isk \textbf{A}ssessment and Mitigatio\textbf{N} (ADRIAN) could come into play. 