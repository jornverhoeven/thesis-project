\section{Timeline} \label{sec:timeline}
Figure \ref{fig::timeline} shows the estimated timeline of the project. The project will be following an Agile approach in which sprints will be used to break up the overall schedule into smaller time-frames. In the timeline, these sprints are indicated by the vertical red lines. 

The rows are divided into four sections: \emph{Learning}. \emph{Code implementation}, \emph{Experiment execution}, \emph{Thesis writing}. These sections help to organize the overall flow of the project.

\begin{description}
    \item[Learning] This section is used to fill any knowledge gaps to ensure the project can be completed successfully. Additional research and case studies are also performed during this time to refine the experiments and investigate potential improvements to the protocol.
    \item[Code Implementation] focuses on the implementation of the ADRIAN protocol in code. After each of the increments (sprints) the code should be ready for simulating the experiment scenarios. 
    \item[Experiment Execution] is a section with tasks pertaining to the execution of the experiment simulations and gathering of the measurements. These experiments help answer the research questions from section \ref{sec:project-summary}.
    \item[Thesis Writing] section holds tasks that will focus on the actual thesis writing. It runs for the entirety of the project as every tasks can be documented from the moment the project starts. At the beginning of each sprint we sit down and finalize the methodology used for the next experiment. At the end of each sprint we use the collected measurements and evaluate the results.
\end{description}

\begin{landscape}

\begin{figure}[ht]

\begin{ganttchart}[
hgrid,
vgrid={*2dotted, red, *3dotted, red, *3dotted, red, *3dotted, red, *3dotted, red},
x unit=.6cm, 
y unit title=0.6cm,
y unit chart=0.6cm,
bar/.append style={fill=black!20},
bar label node/.append style=%
{align=left}
]{1}{18}
\gantttitle{Weeks}{18} \\
\gantttitlelist{20,...,37}{1} \\

\ganttgroup{Learning}{1}{6} \\
\ganttbar{Case-studies for relevant work}{1}{2} \\
\ganttbar{Graph Theory / Knowledge bases}{3}{5} \\

\ganttgroup{Code Implementation}{1}{14} \\
\ganttbar{Core Application Architecture}{1}{5} \\ 
\ganttbar{Knowledge Base}{4}{5} \\ 
\ganttbar{Attack Graph}{6}{7} \\ 
\ganttbar{Risk Assessment}{8}{9} \\
\ganttbar{Auctioning}{12}{13} \\ 

\ganttgroup{Experiment Execution}{4}{17} \\ 
\ganttbar{Initial Experiment Setup}{4}{7} \\ 
\ganttbar{RQ-1}{8}{11} \\ 
\ganttbar{RQ-2}{12}{15} \\ 
\ganttbar{Re-evaluation}{16}{17} \\

\ganttgroup{Thesis writing}{1}{18} \\
\ganttbar{Background \& Related work}{1}{6} \\
\ganttbar{Methodology}{4}{4}
\ganttbar{}{8}{8} 
\ganttbar{}{12}{12} \\
\ganttbar{Results \& evaluation}{7}{7}
\ganttbar{}{11}{11} 
\ganttbar{}{15}{15} \\
\ganttbar{Conclusion}{16}{17} \\
\ganttbar{Fine-tuning}{18}{18}

\end{ganttchart}

\caption{\label{fig::timeline}Thesis timeline}
\end{figure}
\end{landscape}