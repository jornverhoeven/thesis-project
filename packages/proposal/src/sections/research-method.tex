\section{Research Method}
The thesis primarily focuses on implementing and experimenting with the ADRIAN protocol \cite{mann2023ADRIAN}. This protocol is theorized to overcome the problems that were mentioned in the sections before. For this thesis, a Proof of Concept will be created based on this protocol, which will be used to do further validation to answer the research questions from Section~\ref{sec:project-summary}. 
% \todo{Q: Who is 'we' and what does 'think' have to do with the research question? A: Even though I perform the research etc. we still are considered authors of this paper as I've been told. }We think that first creating an attack graph using predefined risk rules should allow any nodes within the network to assess the risk, in a localized portion of the network. We also think that allowing neighboring agents to propose mitigations will reduce the overall risk of the network. 

To initiate, experiment with and refine the PoC we will perform a 
\begin{enumerate}
    \item Literature study
    \item Define hypotheses
    \item PoC Implementation
    \item Experiments
\end{enumerate}

\subsection{Literature Study}
An initial literature study is done to investigate the current bleeding-edge technology and recent literature for risk assessment and mitigation strategies. We apply the knowledge from this literature study together with the ADRIAN protocol to create a proof of concept that combines the hypothesized strengths (Subsection~\ref{ssec:hypothesis}). This proof of concept is then used to evaluate the proposed protocol by running several experiments (Subsection~\ref{ssec:experiments}).

\subsection{Hypothesis} \label{ssec:hypothesis}
As explained in Section \ref{sec:problem-analysis}, existing research has been done to do risk identification and intrusion detection. The provisional hypothesis is that by implementing the ADRIAN protocol, it would be possible to do automated risk assessment and mitigation.
% \todo{It is unclear how the hypotheses relate to the research questions}We hypothesize that by letting agents within a network communicate and negotiate with one another according to the proposed protocol, the overall risk probability of a network will decrease. 

The hypothesis is that implementing the ADRIAN-protocol, in which agents calculate Attack Graphs based on a local knowledge base, allows the network agents to identify risks accurately and efficiently. By tweaking several variables, such as the distance each agent can communicate (See Section \ref{ssec:experiments}), it is possible to analyze the impact of different strategies which are mentioned in the ADRIAN concept \cite{mann2023ADRIAN}. This research aims to investigate the overall effect of this method, and will not be an exhaustive list of possible strategies. 

Additionally, it is hypothesized that implementation of the ADRIAN shows a decrease in the overall risk, after applying adaptation patterns over time.

\subsection{PoC Implementation} \label{ssec:implementation}
In order to validate the hypothesis and answer the research questions a Proof of Concept (PoC) will be implemented as explained in Section \ref{sec:expected-results}. This PoC allows the execution of several experiments in a controlled environment, where different networks and events can be simulated. 

\subsection{Experiments} \label{ssec:experiments}
With the PoC implemented it is possible to run the different experiments. Each experiment is performed over multiple epochs. An epoch is defined as a time window that is initiated with an initial infrastructure, subject to modifications made by agents through adaptations. A property of this epoch is that it does not allow for the introduction of new nodes or software components to the network. If modifications to the network should be considered, they are added to the initial infrastructure of the subsequent epoch.
% We define an epoch as a time window with an initial infrastructure that can be changed by agents by means of adaptations. This initial infrastructure can be different in each subsequent epoch, \todo{Not clear what is meant here}based on these mutations. No new infrastructure nodes or software components are added during an epoch. If mutations to the infrastructure should be considered, they are added to the initial infrastructure of the subsequent epoch.

\subsubsection{Baseline Experiment} 
To evaluate the effectiveness of the ADRIAN protocol we will perform a baseline experiment. This experimental setup allows nodes to apply predefined adaptation patterns to the properties and attributes of the node it is running on, just like in the concept by Mann and Smolka \cite{mann2023ADRIAN}. However, the agents are not allowed to communicate with their neighbors, nor work together on proposing improvements. What this effectively simulates is a network in which agents only identify and mitigate risks on their own hosting node. 

In our baseline environment, we can still calculate the overall risk probability and measure it over time. Thinks to consider when measuring are (but not limited to); the number of adaptations, the total number of risks identified, the number of remaining risks, and the sum of the damage for remaining risks.

% \add{Experiment to verify cooperation on Risk Analysis}
\subsubsection{Experiment 1: Risk Analysis (communication)} 
In this ex'periment, we aim to evaluate the impact and cost of different knowledge exchange ranges. For this experiment, we attempt different ranges (See Section \ref{ssec:range}) and measure the effects over time. Thinks to consider when measuring are (but not limited to); the number of adaptations, the total number of messages exchanged, the total number of risks identified, the number of remaining risks, and the sum of the damage for remaining risks.

% \add{Experiment to verify cooperation on Risk Mitigation}
\subsubsection{Experiment 2: Risk Mitigation (cooperation)}
This third experiment is used to verify the impact and effectiveness of the cooperation of the agents. It will be possible for agents to initiate and participate in auctions, aiming to reduce the overall risk. As with the last experiment, we will measure the number of adaptations, the total number of messages exchanged, the total number of risks identified, the number of remaining risks, and the sum of the damage for remaining risks.


\subsubsection{Broadcasting range}\label{ssec:range}
Agents will be able to communicate with other agents in the network within an exchange range of \( Range(n) \), where \(n\) is a number representing the maximum distance (shortest path) between nodes. In our baseline experiment, we effectively set \( Range(0) \) for the entire network, meaning that no knowledge is shared between nodes. \( Range(1) \) means that knowledge is shared between immediate neighbors. 


% \question{Maybe it would be interesting to investigate the option where an agent can 'vouch' for another node. Sometimes Agents cannot be run on an actual node. With the abstraction it doesn't really matter too much. But from a practical point of view, it might be an interesting concept.}
