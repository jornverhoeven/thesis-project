\section{Methodology}
\label{sec:methodology}
The initial phase of this research involves the development and realization of a POC implementation of the ADRIAN Protocol, of which the latter is designed by Mann and Smolka in \cite{mann2023ADRIAN}. This POC is then used to run multiple simulations in a controlled environment, which allows the generation of metrics as described in Section \ref{sec:experiments}. These metrics, of which the results are in Section \ref{sec:results}, are then used to validate the POC implementation quantitatively. 

% \subsection{Research Approach and Design}
% \label{ssec:research-approach}
% \begin{quote}\textcolor{red}{
%     Describe the overall strategy used to tackle the problem. Is it an experimental, theoretical, or a combination of both? Explain why this approach is appropriate for the research.
% }\end{quote}

\subsection{Research Questions}
\label{ssec:research-questions}

The main research question this thesis will answer is the following question;

\vspace{0.5em}
\textbf{Research Question}\label{rq} \emph{Can the ADRIAN protocol be implemented and used for effective Risk Assessment and Mitigation?}\vspace{1em}

We envision that we can answer this overarching question by extracting two sub-questions from it:

\researchquestion{assessment}{(Identification) Can we use the ADRIAN protocol to do automated risk identification within a network of nodes with an (imperfect) local knowledge base? }

\researchquestion{mitigation}{(Mitigation) Can the ADRIAN protocol be used to decrease the overall risk, by applying adaptation patterns over time?}

% \subsection{Scope and Limitations}
% \label{ssec:scope-limitations}
% \begin{quote}\textcolor{red}{
%     Identify any limitations or potential sources of error in your methodology. This shows that you're aware of the constraints of your approach and helps readers interpret your results appropriately.
% }\end{quote}
