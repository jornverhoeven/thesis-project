\section{Introduction}
\label{sec:introduction}
The exponential growth of the internet has revolutionized several aspects of our modern life, such as communication, entertainment, and the way we work. Not only in the form of computers and phones, but also devices such as smart home assistants, wearables, and industrial sensors. They have enabled us to gather more data and automate many tasks, leading to increased efficiency and convenience. However, this increasing level of connectivity also brings inherent security risks and breaches \cite{khandelwal2016friday, wei2018casino}. These can have harmful effects on individuals and organizations, as they can compromise sensitive information, and cause substantial financial and reputational damage. As a result, there is a growing need for effective and possibly automated risk management and network security measures to mitigate security risks.

In a network of servers and other connected devices, it is often hard to keep track of all the possible vulnerabilities and their impact on the overall security of the network. This is especially true for IoT devices, as they are often not designed with security in mind \cite{miettinen2017iot} and are notoriously hard to update \cite{wurm2016security}. This is a problem, as these devices are often connected to the internet and can be used as a gateway to the rest of the network. 

% \comment{Zoltan}{I find it a bit strange that there is so much focus on IoT. I think the thesis is not specific to IoT}
% \comment{Zoltan}{These are valid problems, but i'm afraid the thesis will not solve them. By describing the problem, you create expectations that you will not fulfill}
% Users of smart devices such as Smart Thermometers, WiFi-connected switches, and IP Cameras are often not aware of any security vulnerabilities and their impact on their privacy and security. Where the online presence of users is receiving more and more focus these days, in the form of Multi-Factor Authentication and strong random passwords through the aid of Password Managers, IoT devices are still seemingly \emph{unprotected}. Most devices have no significant security but passwords that the user never changes. This leaves them vulnerable to a plethora of attacks \cite{hamza2019detecting, paudel2019detecting}. 

Most servers and hardware components have effective firewall settings and security measures, which significantly reduce the potential for unauthorized network access. However, it is important to note that not all devices have settings to prohibit devices from communicating with one another within the same network. Even if these settings are present some of these safeguards are turned off by default. As a result, devices with inadequate protection become easy targets for attackers. In a network of connected devices, this means that the likelihood of an attacker gaining access to a single node becomes higher as vulnerable devices are added to the network.

In an effort to track known vulnerabilities, and ultimately help automate the process of risk identification, the Common Vulnerabilities and Exposures (CVEs) system was created. This list allows software vendors to use static code analysis tools to quickly and cost-effectively find vulnerable pieces of software in their products and mitigate them accordingly. 

% Vulnerabilities that are found are usually registered in the Common Vulnerabilities and Exposures (CVEs), a system for publicly known vulnerabilities. This list allows software vendors to use static code analysis tools to quickly and cost-effectively find vulnerable pieces of software in their products and mitigate them. However, smart home devices are often hard (if not impossible) for end-users to update, leaving the older devices vulnerable to attacks. Some devices have the possibility to actively trigger a firmware update, but more often than not these updates are received in plain text \cite{wurm2016security}. This makes it impossible to guarantee that a smart home device stays secure over time.

In recent years, much research has been done into detecting security risks and intrusions within a network, specifically for IoT devices. Some researchers have investigated if machine learning could prove useful for the task of risks and intrusion detection \cite{canedo2016using, doshi2018machine, hamza2019detecting, sivanathan2018classifying}. This approach seems very accurate to a point where 99 percent of the anomalies in a network could be detected. This seems like a lot, and this accuracy on it's own is quite the achievement, but that single percent that is missing could potentially still do a lot of damage \cite{wei2018casino}. Next to that, these machine-learning models require full access to network packets to properly function which might not always be possible. These network packets would contain information such as protocol, packet size, port numbers, cipher suites, and other detailed information about the traffic. Besides leveraging the power of machine learning more research has been performed to investigate more preventative methods \cite{miettinen2017iot, hamza2019detecting, paudel2019detecting}.

Zarpelao et al. performed a survey to investigate and classify different types of intrusion detection \cite{zarpelao2017survey}. As they mention in their paper, it is not evident which method is best suited for intrusion detection in IoT Systems. This is the stepping stone through which Mann and Smolka want to enter the debate and propose another approach to the problem \cite{mann2023ADRIAN}. 
Mann and Smolka leverage a node of graphs similarly to Paudel et al. \cite{paudel2019detecting}, but instead of inspecting network traffic, they propose looking at the properties of the infrastructure. By using a set of risk rules, which are based on known CVEs, Attack Graphs are created which help to further reason about the risk of a network. More on this in Section \ref{ssec:adrian}.

This research aims to bridge the scientific knowledge gap by implementing the protocol to do \ADRIAN (ADRIAN in short). This protocol, as mentioned above, has been conceptualized by Z.A. Mann and S. Smolka \cite{mann2023ADRIAN}. This research builds upon their ideas and concepts, implementing and experimenting with a Proof of Concept (PoC) to verify and potentially suggest improvements for further research. 

\paragraph{Paper Organization}
Up till this point, we've given a brief introduction to the problem space and the research that has been done in this area. In Section \ref{sec:background} we will briefly discuss the background of this research and the concepts that are used throughout this thesis. In Section \ref{sec:methodology} we will discuss the methodology that has been used to answer the research questions. Section \ref{sec:software-realization} the overview of, and explanation of the implemented architecture will be given. Section \ref{sec:experiments} will give a detailed explanation of the executed experiments, their controls and variables. In Section \ref{sec:results} we go over the results of the experiments that have been performed. In Section \ref{sec:discussion} we will discuss the results and answer the research questions. In Section \ref{sec:conclusion} we will conclude this thesis and discuss future work.
