\section{Introduction}
\label{sec:introduction}
The exponential growth of the internet has revolutionized several aspects of our modern life, such as communication, entertainment, and the way we work. Not only in the form of computers and phones, but also devices such as smart home assistants, wearables, and industrial sensors. They have enabled us to gather more data and automate many tasks, leading to increased efficiency and convenience. However, this increasing level of connectivity also brings inherent security risks and breaches \cite{khandelwal2016friday, wei2018casino}. These can have harmful effects on individuals and organizations, as they can compromise sensitive information, and cause substantial financial and reputational damage. As a result, there is a growing need for effective and possibly automated risk management and network security measures to mitigate security risks.

In a network of servers and other connected devices, it is often difficult to keep track of all the possible vulnerabilities and their impact on the overall security of the network. This is especially true for Internet-of-Things (IoT) devices, as they are often not designed with security in mind \cite{miettinen2017iot} and are notoriously hard to update \cite{wurm2016security}. This is a problem, as these devices are often connected to the internet and can be used as a gateway to the rest of the network. 

% \comment{Zoltan}{I find it a bit strange that there is so much focus on IoT. I think the thesis is not specific to IoT}
% \comment{Zoltan}{These are valid problems, but i'm afraid the thesis will not solve them. By describing the problem, you create expectations that you will not fulfill}
% Users of smart devices such as Smart Thermometers, WiFi-connected switches, and IP Cameras are often not aware of any security vulnerabilities and their impact on their privacy and security. Where the online presence of users is receiving more and more focus these days, in the form of Multi-Factor Authentication and strong random passwords through the aid of Password Managers, IoT devices are still seemingly \emph{unprotected}. Most devices have no significant security but passwords that the user never changes. This leaves them vulnerable to a plethora of attacks \cite{hamza2019detecting, paudel2019detecting}. 

Most servers have (effective) firewall settings and security measures, which significantly reduce the potential for unauthorized network access. However, it is important to note that not all devices have the appropriate settings to prohibit which devices can access it. Even if these settings are present some of these safeguards are turned off by default. As a result, devices with inadequate protection become easy targets for attackers. In a network of connected devices, this means that the likelihood of an attacker gaining access to a single node becomes higher as vulnerable devices are added to the network.

In an effort to track known vulnerabilities, and ultimately help automate the process of risk identification, the Common Vulnerabilities and Exposures (CVEs)\footnote{The official CVE Website can be found here \url{https://www.cve.org/} } system was created. This list allows software vendors to use static code analysis tools to quickly and cost-effectively find vulnerable pieces of software in their products and mitigate them accordingly. 

% Vulnerabilities that are found are usually registered in the Common Vulnerabilities and Exposures (CVEs), a system for publicly known vulnerabilities. This list allows software vendors to use static code analysis tools to quickly and cost-effectively find vulnerable pieces of software in their products and mitigate them. However, smart home devices are often hard (if not impossible) for end-users to update, leaving the older devices vulnerable to attacks. Some devices have the possibility to actively trigger a firmware update, but more often than not these updates are received in plain text \cite{wurm2016security}. This makes it impossible to guarantee that a smart home device stays secure over time.

In recent years, considerable research has been conducted on detecting security risks and intrusions within networks, particularly focussing on IoT devices. Researchers have explored the potential of machine learning in the task of risks and intrusion detection \cite{canedo2016using, doshi2018machine, hamza2019detecting, sivanathan2018classifying}. This approach seems very accurate to a point where 99 percent of the anomalies in a network could be detected. Even though this level of accuracy is quite an achievement, the single percent of inaccuracy could potentially still cause a large amount of damage \cite{wei2018casino}. Next to that, these machine learning models require full access to network packets to properly function which might not always be possible. These network packets would contain information such as protocol, packet size, port numbers, cipher suites, and other detailed information about the traffic. Besides leveraging the power of machine learning more research has been performed to investigate more preventative methods \cite{miettinen2017iot, hamza2019detecting, paudel2019detecting}. Zarpelao et al. conducted a survey to investigate and classify different types of intrusion detection \cite{zarpelao2017survey}. As they mention in their paper, it is not evident which method is best suited for intrusion detection in IoT systems. 

Based on this insight, Mann and Smolka \cite{mann2023ADRIAN} conceptualize the ADRIAN protocol to address the problem. They identified the need for a multi-agent approach in which agents perform risk management in an automated and decentralized way, called \ADRIAN (ADRIAN in short). The agents coordinate their knowledge and actions to detect risks and negotiate on different adaptations strategies that can be autonomously applied to reduce the estimated damage. ADRIAN leverages graphs similarly to Paudel et al. \cite{paudel2019detecting}, but instead of inspecting network traffic, ADRIAN investigates the infrastructure's properties. By using a set of risk rules, which are based on known CVEs, ADRIAN creates attack graphs which help to further reason about the risk of a network. More on this in Section \ref{ssec:adrian}. 

The unpublished concept by Mann and Smolka \cite{mann2023ADRIAN} is only a conceptual solution, and thus lacks a working implementation to evaluate and validate the effectiveness of the solution. It is unclear if this form of cooperation is effective for risk identification, and for automated risk mitigation. This research aims to bridge the scientific knowledge gap by implementing the ADRIAN protocol. This thesis builds upon their ideas and concepts, implementing and experimenting with a Proof of Concept (PoC) to verify and potentially suggest improvements for further research. 

% \comment{Zoltan}{It would be an important aim for the introduction to explain this gap. But this is missing at the moment. I think you should state that Mann \& Smolka identified the need for a multi-agent approach in which agents perform risk management in an automated and decentralized way, and coordinate their knowledge and actions. Also, Mann \& Smolka provided in their unpublished work a conceptual solution for this called ADRIAN, but it is not clear if it really works.}

\subsection*{Research Questions}
\label{ssec:research-questions}
Through the implementation of the ADRIAN concept by Mann and Smolka \cite{mann2023ADRIAN}, it is possible to evaluate and validate the effectiveness of the ADRIAN protocol. By running the software in a multitude of scenarios with different features enabled, we can collect key metrics to quantify the performance of the ADRIAN protocol. This in turn helps to quantify the effectiveness of the cooperative risk identification, and the automated risk mitigation.

This thesis aims to answer the following main research question:

\vspace{0.5em}
\noindent\textbf{Research Question}\label{rq} \emph{Can the ADRIAN protocol be implemented and used for effective Risk Assessment and Mitigation?}\vspace{1em}

We envision that we can answer this overarching question by answering the following sub-questions:

\researchquestion{assessment}{(Identification) Can we use the ADRIAN protocol to do automated risk identification within a network of nodes with an (imperfect) local knowledge base? }

\researchquestion{mitigation}{(Mitigation) Can the ADRIAN protocol be used to decrease the overall risk, by applying adaptation patterns over time?}

\paragraph{Paper Organization}
Up till this point, a brief introduction to the problem space and the research that has been done is provided. In Section \ref{sec:background} the background of this thesis and the concepts that are used throughout this thesis are discussed. Section \ref{sec:software-realization} proposes an overview of, and explanation of the implemented architecture. Section \ref{sec:experiments} will give a detailed explanation of the executed experiments, their control- and independent variables. In Section \ref{sec:results} the results of the experiments that have been performed will be discussed. In Section \ref{sec:discussion} we will discuss the results and answer the research questions. Section \ref{sec:future-research} details some topic for future research. And lastly, in Section \ref{sec:conclusion} we will conclude this thesis and discuss future work.
