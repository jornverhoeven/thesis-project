\section{Future Research}
\label{sec:future-research}

After running the experiments and analyzing the results, we have identified some areas of (potential) improvement in the system. In this section we will discuss some of these areas.

In Section \ref{sssec:knowledge-depth} the concept of knowledge depth was mentioned. As this research kept it at a constant value of 1, it would be interesting to see how this system would perform with different values. This could be done by running the experiments again, but with different values for the knowledge depth. This would give a better understanding of how the knowledge depth affects the performance of the system.

Another area for further research is the risk rules. As mentioned in Section \ref{sssec:risk-rules}, the risk rules are quite simple. It would be interesting to see how the system would perform with more complex risk rules. This could be done by running the experiments again, but with more complex risk rules. This would give a better understanding of how the risk rules affect the performance of the system. 

During the implementation and experimentation some back-and-forth discussions were held about the risk reports that were sent during an auction. The risk reports contain a subsection of the entire attackgraph a node calculated, containing only the nodes that are part of the risk path the auction tries to mitigate. This subsection would contain all risk edges between the nodes, including their probabilities (See figure \ref{fig:riskreport-a}). This would result in a large amount of data being sent during an auction. 

\begin{figure}[H]
    \centering
    \begin{subfigure}[b]{0.4\textwidth}
        \centering
        \includegraphics[width=\textwidth]{_content/riskreport-future-research-a.png}
        \caption{Full risk report with all edges.}
        \label{fig:riskreport-a}
    \end{subfigure}
    \hspace{0.5cm}
    \begin{subfigure}[b]{0.4\textwidth}
        \centering
        \includegraphics[width=\textwidth]{_content/riskreport-future-research-b.png}
        \caption{Merged risk report with only one edge.}
        \label{fig:riskreport-b}
    \end{subfigure}
    \caption{Two different methods of dealing with risk reports, where on the left all risk edges are present and no information is lost. On the right all risk edges between nodes have been merged into a single edge, resulting in a smaller amount of data being sent.}
\end{figure}


An alternative to sending all edges, was to merge the probabilities of all the edges between two nodes into a single edge (See figure \ref{fig:riskreport-b}). This merging could be done similar to the logic that is used when calculating the risk damage value, for example $p = \prod_{k=1}^{R} (1 - p_{k})$ where $R$ is the risk edges between the nodes, and $p_{k}$ is the probability for an edge. 
This would result in a smaller amount of data being sent during an auction. However, this would also result in a loss of information. Since there are no actual messages sent over the network in the experiments, the amount of data that was sent was not a bottleneck, so the decision was made to send all edges of the subgraph. However, if this system were to be implemented in a real-world scenario, the amount of data sent could be a bottleneck. In that case, it would be interesting to see how the system would perform with the alternative risk report.

