\section*{Abstract}
With the ever growing size of \textit{the cloud} we are now, more than ever, in need of systems that can help keep our infrastructure and data secure. In this thesis we propose an implementation of a multi-agent system that is able to detect risks and negotiate on different adaptations strategies that can be autonomously applied to reduce the estimated damage. Through a software realization of the ADRIAN protocol, we show that the agents are able to keep the damage of the infrastructure at a low level, while also keeping the time spent adapting low. By running the software in a multitude of scenarios with different features enabled we can quantify the performance of the ADRIAN protocol. The implementation of, and the ADRIAN protocol itself, both have some limitations that are discussed in this thesis. But overall it can be concluded from this research that the ADRIAN protocol has great potential to be used in real-world scenarios.

\comment{Zoltan}{It is not clear how the size of the cloud (whatever that means) motivates your work. Try to be more specific: motivate why we need a multi-agent approach to security risk management.}
\comment{Sven}{This sentence should be preceded by one or two senteces that motivate why risk mitigation should be realized with multi-agent systems (what are the advantages of realizing risk mitigation with multi-agent systems).}
\comment{Zoltan \& Sven}{You should briefly introduce it (ADRIAN protocol), otherwise readers will have no idea what you are talking about.}
\comment{Zoltan}{It would be good to give here some highlights of the results.}
\comment{Sven}{I would not mention in the abstract that the protocol has limitations. You ahve to sell your idea convincingly in the abstract and in the introduction. Better Idea: Say that the ADRIAN implementations ahve delivered promising results. at the same time, you can also directly state a percentual number, e.g. how well ADRAIN was able to imrpove something/ how many risks it was able to mitigate.}
\comment{Zoltan}{Avoid 'We'}
